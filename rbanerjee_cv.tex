%%%%%%%%%%%%%%%%%%%%%%%%%%%%%%%%%%%%%%%%%%%%%%%%%%%%%%%%%%%%%%%%%%%%%%%%
%%%%%%%%%%%%%%%%%%%%%% Simple LaTeX CV Template %%%%%%%%%%%%%%%%%%%%%%%%
%%%%%%%%%%%%%%%%%%%%%%%%%%%%%%%%%%%%%%%%%%%%%%%%%%%%%%%%%%%%%%%%%%%%%%%%

%%%%%%%%%%%%%%%%%%%%%%%%%%%%%%%%%%%%%%%%%%%%%%%%%%%%%%%%%%%%%%%%%%%%%%%%
%% NOTE: If you find that it says                                     %%
%%                                                                    %%
%%                           1 of ??                                  %%
%%                                                                    %%
%% at the bottom of your first page, this means that the AUX file     %%
%% was not available when you ran LaTeX on this source. Simply RERUN  %%
%% LaTeX to get the ``??'' replaced with the number of the last page  %%
%% of the document. The AUX file will be generated on the first run   %%
%% of LaTeX and used on the second run to fill in all of the          %%
%% references.                                                        %%
%%%%%%%%%%%%%%%%%%%%%%%%%%%%%%%%%%%%%%%%%%%%%%%%%%%%%%%%%%%%%%%%%%%%%%%%

%%%%%%%%%%%%%%%%%%%%%%%%%%%% Document Setup %%%%%%%%%%%%%%%%%%%%%%%%%%%%

% Don't like 10pt? Try 11pt or 12pt
\documentclass[10pt]{article}

% This is a helpful package that puts math inside length specifications
\usepackage{calc}


% Simpler bibsection for CV sections
% (thanks to natbib for inspiration)
\makeatletter
\newlength{\bibhang}
\setlength{\bibhang}{1em}
\newlength{\bibsep}
 {\@listi \global\bibsep\itemsep \global\advance\bibsep by\parsep}
\newenvironment{bibsection}%
        {\vspace{-\baselineskip}\begin{list}{}{%
       \setlength{\leftmargin}{\bibhang}%
       \setlength{\itemindent}{-\leftmargin}%
       \setlength{\itemsep}{\bibsep}%
       \setlength{\parsep}{\z@}%
        \setlength{\partopsep}{0pt}%
        \setlength{\topsep}{0pt}}}
        {\end{list}\vspace{-.6\baselineskip}}
\makeatother

% Layout: Puts the section titles on left side of page
\reversemarginpar

%
%         PAPER SIZE, PAGE NUMBER, AND DOCUMENT LAYOUT NOTES:
%
% The next \usepackage line changes the layout for CV style section
% headings as marginal notes. It also sets up the paper size as either
% letter or A4. By default, letter was used. If A4 paper is desired,
% comment out the letterpaper lines and uncomment the a4paper lines.
%
% As you can see, the margin widths and section title widths can be
% easily adjusted.
%
% ALSO: Notice that the includefoot option can be commented OUT in order
% to put the PAGE NUMBER *IN* the bottom margin. This will make the
% effective text area larger.
%
% IF YOU WISH TO REMOVE THE ``of LASTPAGE'' next to each page number,
% see the note about the +LP and -LP lines below. Comment out the +LP
% and uncomment the -LP.
%
% IF YOU WISH TO REMOVE PAGE NUMBERS, be sure that the includefoot line
% is uncommented and ALSO uncomment the \pagestyle{empty} a few lines
% below.
%

%% Use these lines for letter-sized paper
\usepackage[paper=letterpaper,
            %includefoot, % Uncomment to put page number above margin
            marginparwidth=1.2in,     % Length of section titles
            marginparsep=.05in,       % Space between titles and text
            margin=1in,               % 1 inch margins
            includemp]{geometry}

%% Use these lines for A4-sized paper
%\usepackage[paper=a4paper,
%            %includefoot, % Uncomment to put page number above margin
%            marginparwidth=30.5mm,    % Length of section titles
%            marginparsep=1.5mm,       % Space between titles and text
%            margin=25mm,              % 25mm margins
%            includemp]{geometry}

%% More layout: Get rid of indenting throughout entire document
\setlength{\parindent}{0in}

%% This gives us fun enumeration environments. compactitem will be nice.
\usepackage{paralist}

%% Reference the last page in the page number
%
% NOTE: comment the +LP line and uncomment the -LP line to have page
%       numbers without the ``of ##'' last page reference)
%
% NOTE: uncomment the \pagestyle{empty} line to get rid of all page
%       numbers (make sure includefoot is commented out above)
%
\usepackage{fancyhdr,lastpage}
\pagestyle{fancy}
%\pagestyle{empty}      % Uncomment this to get rid of page numbers
\fancyhf{}\renewcommand{\headrulewidth}{0pt}
\fancyfootoffset{\marginparsep+\marginparwidth}
\newlength{\footpageshift}
\setlength{\footpageshift}
          {0.5\textwidth+0.5\marginparsep+0.5\marginparwidth-2in}
\lfoot{\hspace{\footpageshift}%
       \parbox{4in}{\, \hfill %
                    \arabic{page} of \protect\pageref*{LastPage} % +LP
%                    \arabic{page}                               % -LP
                    \hfill \,}}

% Finally, give us PDF bookmarks
\usepackage{color,hyperref}
\definecolor{darkblue}{rgb}{0.0,0.0,0.3}
\hypersetup{colorlinks,breaklinks,
            linkcolor=darkblue,urlcolor=darkblue,
            anchorcolor=darkblue,citecolor=darkblue}

%%%%%%%%%%%%%%%%%%%%%%%% End Document Setup %%%%%%%%%%%%%%%%%%%%%%%%%%%%


%%%%%%%%%%%%%%%%%%%%%%%%%%% Helper Commands %%%%%%%%%%%%%%%%%%%%%%%%%%%%

% The title (name) with a horizontal rule under it
%
% Usage: \makeheading{name}
%
% Place at top of document. It should be the first thing.
\newcommand{\makeheading}[1]%
        {\hspace*{-\marginparsep minus \marginparwidth}%
         \begin{minipage}[t]{\textwidth+\marginparwidth+\marginparsep}%
                {\large \bfseries #1}\\[-0.15\baselineskip]%
                 \rule{\columnwidth}{1pt}%
         \end{minipage}}

% The section headings
%
% Usage: \section{section name}
%
% Follow this section IMMEDIATELY with the first line of the section
% text. Do not put whitespace in between. That is, do this:
%
%       \section{My Information}
%       Here is my information.
%
% and NOT this:
%
%       \section{My Information}
%
%       Here is my information.
%
% Otherwise the top of the section header will not line up with the top
% of the section. Of course, using a single comment character (%) on
% empty lines allows for the function of the first example with the
% readability of the second example.
\renewcommand{\section}[2]%
        {\pagebreak[2]\vspace{1.3\baselineskip}%
         \phantomsection\addcontentsline{toc}{section}{#1}%
         \hspace{0in}%
         \marginpar{
         \raggedright \scshape #1}#2}

% An itemize-style list with lots of space between items
\newenvironment{outerlist}[1][\enskip\textbullet]%
        {\begin{itemize}[#1]}{\end{itemize}%
         \vspace{-.6\baselineskip}}

% An environment IDENTICAL to outerlist that has better pre-list spacing
% when used as the first thing in a \section
\newenvironment{lonelist}[1][\enskip\textbullet]%
        {\vspace{-\baselineskip}\begin{list}{#1}{%
        \setlength{\partopsep}{0pt}%
        \setlength{\topsep}{0pt}}}
        {\end{list}\vspace{-.6\baselineskip}}

% An itemize-style list with little space between items
\newenvironment{innerlist}[1][\enskip\textbullet]%
        {\begin{compactitem}[#1]}{\end{compactitem}}

% An environment IDENTICAL to innerlist that has better pre-list spacing
% when used as the first thing in a \section
\newenvironment{loneinnerlist}[1][\enskip\textbullet]%
        {\vspace{-\baselineskip}\begin{compactitem}[#1]}
        {\end{compactitem}\vspace{-.6\baselineskip}}

% To add some paragraph space between lines.
% This also tells LaTeX to preferably break a page on one of these gaps
% if there is a needed pagebreak nearby.
\newcommand{\blankline}{\quad\pagebreak[2]}

% Uses hyperref to link DOI
\newcommand\doilink[1]{\href{http://dx.doi.org/#1}{#1}}
\newcommand\doi[1]{doi:\doilink{#1}}


%%%%%%%%%%%%%%%%%%%%%%%% End Helper Commands %%%%%%%%%%%%%%%%%%%%%%%%%%%

%%%%%%%%%%%%%%%%%%%%%%%%% Begin CV Document %%%%%%%%%%%%%%%%%%%%%%%%%%%%

\begin{document}
\makeheading{Rahul~Banerjee}

\section{Contact Information}
%
% NOTE: Mind where the & separators and \\ breaks are in the following
%       table.
%
% ALSO: \rcollength is the width of the right column of the table
%       (adjust it to your liking; default is 1.85in).
%
\newlength{\rcollength}\setlength{\rcollength}{2.3in}%
%
\begin{tabular}[t]{@{}p{\textwidth-\rcollength}p{\rcollength}}
AC101 Paul Allen Center, Box 352350,  & \textit{Phone:} (425)-894-5701\\
185 Stevens Way, Seattle, WA 98195-2350 & \textit{Mail:} \href{mailto:banerjee@cs.washington.edu}{banerjee@cs.washington.edu}\\
\end{tabular}

\section{Research Interests}
%
Enabling non-experts to create 3D content, Real-time rendering software and hardware.
Current research (with Kathleen Tuite): \href{http://photocitygame.com/pointcraft}{http://photocitygame.com/pointcraft/}

\section{Education}
%
        \href{http://www.cs.washington.edu/}{\textbf{The University of Washington}}, Seattle \hfill \textbf{Sep 2011 -- current}\\
        Ph.D, \emph{Computer Science}

\blankline

        \href{http://www.cse.iitk.ac.in/}{\textbf{The Indian Institute of Technology}}, Kanpur \hfill \textbf{Jul 2003 -- May 2005}\\
        Master of Technology, \emph{Computer Science}

\blankline

        \href{http://www.klyuniv.ac.in/}{\textbf{University of Kalyani}}, Kalyani \hfill \textbf{Aug 1999 -- Jul 2003}\\
        Bachelor of Technology, \emph{Information Technology}

\section{Publication} \begin{bibsection}
    \item Banerjee, R., and A.~Mukherjee. Animating Hand Behaviours Using Virtual Sensors and an Automata Hierarchy. In: \emph{Proceedings of the Fourth Asian Conference on Industrial Automation and Robotics (ACIAR-05)}, May 11-13, 2005, Bangkok, Thailand.
\end{bibsection}

\section{Professional Experience}
%
\href{http://www.dreamworksanimation.com/}{\textbf{DreamWorks}}/\href{http://www.technicolor.com/}{\textbf{Technicolor}},
Bangalore, India \hfill \textbf{Nov 2009 -- May 2011}\\
\textit{Pipeline Technical Director}
\begin{outerlist}
\item[] \textit{Graphics Software:}
\begin{innerlist}
\item Parallelized fur collision simulator to scale better on render farm.
\item Created Maya plugin system for artists to interactively preview fur while editing.\\
      System used for a Hollywood full-length feature and its accompanying DVD special (see ``Other'' section).
\end{innerlist}

\item[] \textit{Key Responsibilities:}
\begin{innerlist}
\item Day-to-day support for 14 artists doing cloth, fur, and contact finaling.
\item Root-causing and fixing geometry/rendering artifacts on production shots.
\item Writing scripts to streamline artists' workflow.
\end{innerlist}
\end{outerlist}

\blankline

\href{http://www.nvidia.com/}{\textbf{NVIDIA}},
Bangalore, India \hfill \textbf{Jul 2005 -- Nov 2009}\\
\textit{Senior Systems Software Engineer}
\begin{outerlist}

\item[] \href{http://www.nvidia.com/object/cuda_opencl_new.html}{\textit{World's first}}~\textit{conformant implementation of}~\href{http://www.khronos.org/opencl/}{\textit{OpenCL 1.0}}:
\begin{innerlist}
\item Part of Khronos OpenCL discussions from initial draft through 1.0 spec stage.
\item Wrote over half of the built-in vector math routines; passed Khronos' test suite.
\item Part of four-member team which implemented code generation backend for \href{http://www.nvidia.com/content/CUDA-ptx_isa_1.4.pdf}{PTX}.
\item Helped port CUDA \href{http://www.youtube.com/watch?v=r1sN1ELJfNo}{gravitational N-body demo} to run on OpenCL, shown at SIGGRAPH Asia 2008 Course:\\
         \href{http://www.siggraph.org/asia2008/attendees/courses/15.php}{``Parallel Computing for Graphics: Beyond Programmable Shading''}.
\end{innerlist}

\item[] \href{http://www.khronos.org/opengles/2_X/}{\textit{OpenGL-ES 2.0}}~\textit{driver for NVIDIA's GeForce 8 GPU:}
\begin{innerlist}
\item Coded blit engine, shader linker, timing and synchronization.
\end{innerlist}

\item[] \textit{H.264 video encoder:}
\begin{innerlist}
\item Implemented and optimized deblocker for custom embedded processor.
\end{innerlist}

\item[] \textit{Systems Software:}
\begin{innerlist}
\item Optimized game engine performance for a client.
\item Wrote kernel drivers, maintained customised toolchains for embedded hardware.
\end{innerlist}

\end{outerlist}

\section{Invited Talk}
%
An Introduction to the 3D Graphics Pipeline, \textit{Ninth International Symposium on Spatial Media (ISSM '08-'09)}, Feb 17-18, 2009, University of Aizu, Japan.

\section{Internship}
%
        \href{http://www.isical.ac.in/}{\textbf{The Indian Statistical Institute}}, Kolkata \hfill \textbf{Jan 2003 -- Apr 2003}\\
                Detecting shapes in images via Fuzzy Hough Transform.
                
\section{Teaching Experience}
%
\href{http://www.dreamworksanimation.com/}{\textbf{DreamWorks}}/\href{http://www.technicolor.com/}{\textbf{Technicolor}},
Bangalore, India \hfill \textbf{2010 -- 2011}\\
  \textit{Training technical staff and artists:}
  \begin{innerlist}
    \item The 3D Rendering Pipeline~[technical training]
    \item Maths for Computer Graphics~[technical training]
    \item Introduction to DreamWorks Pipeline~[artist training]
  \end{innerlist}

\blankline

\href{http://www.nvidia.com/}{\textbf{NVIDIA}},
Bangalore, India \hfill \textbf{Jul 2008 -- Aug 2008}\\
  \textit{Mentoring senior-year undergrad interns:}
  \begin{innerlist}
    \item Fractal rendering on GPU
    \item Plugins to adapt Eclipse IDE for custom hardware
  \end{innerlist}

\blankline

\href{http://www.cse.iitk.ac.in/}{\textbf{The Indian Institute of Technology}}, Kanpur\\
  \textit{Teaching Assistant for undergrad courses:}
  \begin{innerlist}
    \item \href{http://www.cse.iitk.ac.in/teaching/courses/CS425.html}
      {CS~425}: Computer Networks \hfill \textbf{Fall 2003}
    \item \href{http://www.cse.iitk.ac.in/teaching/courses/CS425.html}
      {CS~672}: Natural Language Processing Semantics \hfill \textbf{Spring 2004}
  \end{innerlist}

\section{Technical \\ Skills}
%
Five years of industry experience in real-time and cinematic 3D graphics, ranging from studio pipeline applications and systems software through graphics drivers and GPUs. \\

\textit{Languages:} C, C$+$$+$, Python. \\

\textit{Tools}: gcc, make, gdb, UNIX shell scripting, vim, \LaTeX{}. \\

\textit{Architectures:} x86, x86-64, PowerPC, NVIDIA GPU (GeForce 6, GeForce 8, GeForce 9). \\

\textit{Operating Systems:} Linux, Windows, Mac OS X (10.5).

\section{Other}
%
Founding member, \href{http://chapters.siggraph.org/acm_rss_pc.php}{Bangalore ACM SIGGRAPH Chapter}\\
(Officially chartered Oct 23, 2009)

\blankline

Scholarship supporting 2 years of study at IIT Kanpur\\
(\href{http://www.scholarships-india.com/MHRD_scholarship_assistantship.html}{Ministry of Human Resource Development, India}).

\blankline

Secured 99.38\% percentile score (All-India Rank 227 out of 37,797 candidates) in nation-wide entrance test for admission to Masters programs in Engineering, 2003.

\blankline

Screen credits on
\begin{itemize}
\item{} \href{http://www.imdb.com/title/tt0448694/}{``Puss In Boots''}(Oscar-nominated full-length feature animation, 2011)
\item{} \href{http://www.imdb.com/title/tt2271551/}{``The Three Diablos''}(DVD, 2012)
\item{} \href{http://www.imdb.com/title/tt1725156/}{``Scared Shrekless''}(Television, 2010)
\end{itemize}
\end{document}

%%%%%%%%%%%%%%%%%%%%%%%%%% End CV Document %%%%%%%%%%%%%%%%%%%%%%%%%%%%%
